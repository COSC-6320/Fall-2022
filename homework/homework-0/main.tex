\documentclass[final]{article}
\usepackage{main}

\date{}
\title{Homework 0}
\author{Gopal Pandurangan}

\begin{document}

\maketitle

\begin{tcolorbox}
    \noindent\textbf{Instructions:}

    Read the Academic honesty policy posted on Blackboard which is repeated here:
    All submitted work should be  your own. Copying or using other people's work (including  from the Web) will result in $-\text{MAX}$ points,
    where $\text{MAX}$ is the maximum possible number of points for that assignment. Repeat offense will result in getting a failure grade in
    the course and reporting to the Chair. If you have any questions regarding any assignment, please contact me. The best way is to ask on Piazza.\\

    By submitting this homework, \textbf{you affirm that you have followed the Academic Dishonest Policy.}\\

    \textbf{Justify} your answers Show \textbf{appropriate work}.\\

    Write \textbf{clearly} and \textbf{legibly}.\\

    Please \textbf{start to work on the problems early} as these generally require some time.\\

    We recommend the use of \LaTeX\ for typing solutions. \LaTeX\  is a software system for typesetting documents and has found widespread use
    in academia and is considered the standard way to create mathematical documents. To get started using \LaTeX, the use of
    \href{https://www.overleaf.com}{Overleaf} is recommended. A short tutorial on the use of \LaTeX\ can be found
    \href{https://www.overleaf.com/learn/latex/Learn_LaTeX_in_30_minutes}{here}.\\

    \begin{tcolorbox}[colback=red!5!white,colframe=red!75!black]
        Out: Tuesday, August 23

        Due: Sunday, September 4, 11:59PM

        \textbf{Submissions will not be accepted after the deadline.}
    \end{tcolorbox}

    The exercises below are from the book available at \href{https://sites.google.com/site/gopalpandurangan/home/algorithms-course}{my homepage}.
    Please always use the latest version of the book which is posted at this site, since the book is updated periodically.

    \textbf{Reading:} Chapters 2-4, Appendices A, B, and E. In particular, several worked exercises with solutions are given. Trying to solve
    the worked exercises \textbf{before} seeing their solutions is a good learning technique.

    \begin{tcolorbox}[colback=blue!5!white,colframe=blue!75!black]
        \textbf{Exercises refer to the corresponding chapter exercises in the book}.

        2.4 (part c), 3.4 (part d), 4.1 (part a), 4.3 (part 6), 4.8, Exercises  A.6 (in Appendix A), Exercise B.6 (in Appendix B).
    \end{tcolorbox}
\end{tcolorbox}
\newpage
\begin{exercise}{2.4 (c)}
    Rank the following functions by order of growth. That is,
    find an arrangement $f_1$, $f_2$, $\dots$, $f_k$ of the functions satisfying
    \begin{align*}
        f_1       & = \bigO{f_2}       \\
        f_2       & = \bigO{f_3}       \\
                  & \phantom{..}\vdots \\
        f_{k - 1} & = \bigO{f_k}
    \end{align*}
    Justify your ordering.
    Note that $\log^k{n}$ is the usual way of writing $(\log n)^k$.
    \[n^3,\, \frac{n}{\log^2 n},\, n \log n, 1.1^n,\,
        \frac{1}{n^3},\, \log^{6} n,\, \frac{1}{n},\,
        2^{\lg n},\, n!,\, n^{\lg {\lg n}},\, 2^{\sqrt{\log n}},\,
        n^{\frac{1}{\log n}}\]
\end{exercise}

\begin{solution}
    The ordering is
    \[\frac{1}{n^3}, \frac{1}{n}, n^{\frac{1}{\log{n}}}, \log^6{n}, 2^{\sqrt{\log{n}}}, \frac{n}{\log^2{n}}, 2^{\log{n}}, n\log{n}, n^3, n^{\lg\lg{n}}, 1.1^n, n!\]
    Simply take the limits of these functions in order and show that they all exist and are finite:
    \begin{itemize}
        \item \(\frac{1}{n^3} = \bigO{\frac{1}{n}}\):
              \begin{align*}\lim_{n\to\infty}\frac{\frac{1}{n^3}}{\frac{1}{n}}
                   & = \lim_{n\to\infty}\frac{n}{n^3} \\
                   & = \lim_{n\to\infty}\frac{1}{n^2} \\
                   & = 0
              \end{align*}
        \item \(\frac{1}{n} = \bigO{n^{\frac{1}{\log{n}}}}\) (note that \(n^{\frac{1}{\log{n}}} = 2\)):
              \begin{align*}\lim_{n\to\infty}\frac{\frac{1}{n}}{2}
                   & = \lim_{n\to\infty}\frac{1}{2n} \\
                   & = 0
              \end{align*}

        \item \(n^{\frac{1}{\log{n}}} =\bigO{\log^6{n}}\) (note that \(n^{\frac{1}{\log{n}}} = 2\)):
              \begin{align*}\lim_{n\to\infty}\frac{2}{\log^6{n}}
                   & = 0
              \end{align*}
        \item \(\log^6{n} = \bigO{2^{\sqrt{\log{n}}}}\). We show that the limit of their logs is 0:
              \begin{align*}\lim_{n\to\infty}\frac{\log\left(\log^6{n}\right)}{\log\left(2^{\sqrt{\log{n}}}\right)}
                   & = \lim_{n \to \infty}\frac{6\log\log{n}}{\sqrt{\log{n}}}                             \\
                   & = \lim_{n \to \infty}\frac{\frac{6}{n\ln{2}\ln{n}}}{\frac{1}{2n\sqrt{\ln{2}\ln{n}}}} \\
                   & = \lim_{n \to \infty}\frac{12}{\sqrt{\ln{2}\ln{n}}}                                  \\
                   & = 0
              \end{align*}
        \item \(2^{\sqrt{\log{n}}} = \bigO{\frac{n}{\log^2{n}}}\). We show that the limit of their logs is 0:
              \begin{align*}\lim_{n\to\infty}\frac{\log\left(2^{\sqrt{\log{n}}}\right)}{\log\left(\frac{n}{\log^2{n}}\right)}
                   & = \lim_{n \to \infty}\frac{\sqrt{\log{n}}}{\log{n} - 2\log\log{n}} \\
                   & = 0
              \end{align*}
        \item \(\frac{n}{\log^2{n}} = \bigO{2^{\log{n}}}\) (note that \(2^{\log{n}} = n\)):
              \begin{align*}\lim_{n\to\infty}\frac{\frac{n}{\log^2{n}}}{n}
                   & = \lim_{n\to\infty}\frac{1}{\log^2{n}} \\
                   & = 0
              \end{align*}
        \item \(2^{\log{n}} = \bigO{n\log{n}}\) (note that \(2^{\log{n}} = n\)):
              \begin{align*}\lim_{n\to\infty}\frac{n}{n\log{n}}
                   & = \lim_{n\to\infty}\frac{1}{\log{n}} \\
                   & = 0
              \end{align*}
        \item \(n\log{n} = \bigO{n^3}\):
              \begin{align*}\lim_{n \to \infty}\frac{n\log{n}}{n^3}
                   & = \lim_{n \to \infty}\frac{\log{n}}{n^2}          \\
                   & = \lim_{n \to \infty}\frac{\frac{1}{n\ln{2}}}{2n} \\
                   & = \lim_{n \to \infty}\frac{1}{2n^2\ln2}           \\
                   & = 0
              \end{align*}
        \item \(n^3 = \bigO{n^{\lg\lg{n}}}\). We show that the limit of their logs is 0:
              \begin{align*}\lim_{n\to\infty}\frac{\log\left(n^3\right)}{\log\left(n^{\lg\lg{n}}\right)}
                   & = \lim_{n \to \infty}\frac{3\log{n}}{\lg\lg{n}\log{n}} \\
                   & = \lim_{n \to \infty}\frac{3}{\lg\lg{n}}               \\
                   & = 0
              \end{align*}
        \item \(n^{\lg\lg{n}} = \bigO{1.1^n}\). We show that the limit of their logs is 0:
              \begin{align*}\lim_{n\to\infty}\frac{\log\left(n^{\lg\lg{n}}\right)}{\log\left(1.1^n\right)}
                   & = \lim_{n \to \infty}\frac{\lg\lg{n}\log{n}}{n\log{1.1}} \\
                   & = 0
              \end{align*}
        \item \(1.1^n = \bigO{n!}\). A basic induction proof works here. Notice that \(1.1^2 < 2!\) (this is our base case). Suppose we have that \(1.1^n < n!\) for some \(n\). Then
              \begin{align*}1.1^{n + 1}
                   & = 1.1^n \times 1.1 \\
                   & \leq n! \times 1.1 \\
                   & < n! \times n      \\
                   & = (n + 1)!
              \end{align*}
              A more elegant proof that applies for any \(a > 1\) shows that \(a^n < n!\) for all sufficiently large \(n\). Consider the series \[\sum_{n=1}^{\infty}\frac{a^n}{n!}\] And notice that
              \begin{align*}\lim_{n \to \infty}\frac{\frac{a^{n + 1}}{(n + 1)!}}{\frac{a^n}{n!}}
                   & = \lim_{n \to \infty}\frac{a}{n = 1} \\
                   & = 0
              \end{align*} Thus, by the ratio test, this series converges. In particular, this forces \[\lim_{n \to \infty}\frac{a^n}{n!} = 0\]
    \end{itemize}
\end{solution}

\begin{exercise}{3.4 (d)}
    Prove by induction that $3^n < n!$ for $n \geq 7$.
\end{exercise}

\begin{solution}
    Proceed by induction on $n$. The base case is trivial:
    \[3^7 = 2187 < 5040 = 7!\]

    Suppose that $3^{n - 1} < (n - 1)!$. Then
    \begin{align*}3^n
         & = 3 \times 3^{n - 1}                                    \\
         & < 3 \times (n - 1)! \hbox{ by the induction hypothesis} \\
         & < n \times (n - 1)! \hbox{ since $n \geq 7$, $3 < n$}   \\
         & = n!
    \end{align*}
    as desired.
\end{solution}

\begin{exercise}{4.1 (a)}
    Prove that the asymptotic bound for the following recurrence

    \[T(n) \leq 2 T(n/2) + n^2\]

    is

    \[T(n) = \bigO{n^2 \log n}\]

    by induction. Assume the base cases of the recurrence are
    constants i.e., $T(n) < k$ for $n < n_0$ where $n_0$ and $k$ are some
    constants.
\end{exercise}

\begin{solution}
    Begin by induction on $n$. The base case is given. Our induction hypothesis
    is that, for all $n_0 \leq k < n$, $T(k) \leq ck^2\log{k}$ for some $c > 0$.
    Then
    \begin{align*}T(n)
         & \leq 2T\left(\frac{n}{2}\right) + n^2 \hbox{ by definition of $T(n)$}                      \\
         & \leq 2c\left(\frac{n}{2}\right)^2\log\frac{n}{2} + n^2 \hbox{ by our induction hypothesis} \\
         & = 2c\frac{n^2}{4}\log\frac{n}{2} + n^2                                                     \\
         & = 2c\frac{n^2}{4}\log{n} - 2c\frac{n^2}{4}\log{2} + n^2                                    \\
         & = c\frac{n^2}{2}\log{n} - c\frac{n^2}{2} + n^2                                             \\
         & = c\frac{n^2}{2}\log{n} + n^2\left(-\frac{c}{2} + 1\right)                                 \\
         & \leq c\frac{n^2}{2}\log{n}\hbox{ for $c \geq 2$}                                           \\
         & < cn^2\log{n}\hbox{ for all c \geq 2}\qedhere                                              \\
    \end{align*}
\end{solution}

\begin{exercise}{4.3 (6)}
    Solve the following recurrence

    \[T(n) = 4 T\left(\sfrac{n}{2}\right) + n^3\]

    Give the answer in terms of \emph{Big-Theta} notation. Solve up to constant
    factors, i.e., your answer must give the correct function for $T(n)$, up to constant factors.
    You can assume constant base cases, i.e., $T(1) = T(0) = c$, where $c$ is a positive constant.
    You can ignore floors and ceilings. You can use the DC Recurrence Theorem if it applies.
\end{exercise}

\begin{solution}
    In this case, $a = 4$, $b = 2$, and $f(n) = n^3$. Then
    \begin{align*}af\left(\frac{n}{b}\right)
         & = 4\left(\frac{n}{2}\right)^3 \\
         & = 4\frac{n^3}{8}              \\
         & = \frac{1}{2}n^3
    \end{align*}
    hence $c = \sfrac{1}{2} < 1$ and, by the DC Recurrence Theorem, $T(n) = \bigTh{n^3}$.
\end{solution}

\begin{exercise}{4.8}
    Give a recursive algorithm to compute $b^n$ for a given integer $n$.
    Your algorithm should perform only $\bigO{\log{n}}$ integer multiplications.
\end{exercise}

\begin{solution}
    Notice that, if \(n = b_0b_1\hdots b_k\) is the binary representation of \(n\), then \(2^n=2^{b_0b_1\hdots b_k}=2^{2^k b_0}2^{2^{k - 1}b_1}\cdots 2^{b_n}\). Whenever these \(b_i\) is 0, \(2^{2^{k - i} b_i} = 1\).
    \begin{algorithm}[H]
        \renewcommand{\thealgorithm}{}
        \begin{algorithmic}[1]
            \caption{\texttt{pow}(\(b, n\)): Returns \(b^n\)}
            \Procedure{pow}{b, n}
            \If{\(n = 0\)}
            \State \Return 1
            \Else
            \If{\(n\) is even}
            \State \Return $\Call{pow}{b^2, \frac{n}{2}}$
            \Else
            \State \Return $b \times \Call{pow}{b^2, \floor{\frac{n}{2}}}$
            \EndIf
            \EndIf
            \EndProcedure
        \end{algorithmic}
    \end{algorithm}

    Note that this takes \(\bigO{\log{n}}\) steps, as each step we divide \(n\) by 2, and thus the algorithm terminates in $\floor{\log{n}}$ steps.
\end{solution}

\begin{exercise}{A.6}
    Consider the set of first $2n$ positive integers, i.e., $A = \set{1, 2, \dots, 2n}$. Take any subset $S$ of $n+1$ distinct numbers from set $A$. Show that there  two numbers in $S$ such that one divides the other.
\end{exercise}

\begin{solution}
    Partition $A$ into sets of the form $S_m = \set{m, 2m, 4m, \dots}$, where $m$ is
    odd. Notice that this forms a partition of $\set{1, 2, \dots, 2n}$\footnote{A proof of this is fairly
        straightforward, so we leave it to the reader.} and that, for any $a, b$ in
    a subset $S_m$, we must have $a \divides b$ or $b \divides a$. The number of
    such sets is exactly the number of odd numbers between $1$ and $2n$, which
    is $n$. Thus, by the pigeonhole principle, given $n + 1$ numbers, at least
    two must fall in the same subset $S_m$, and we are done.
\end{solution}

\begin{exercise}{B.6}
    A \emph{strange} number is one whose only prime factors are in the set $\set{2, 3, 5}$.
    Give an efficient algorithm (give pseudocode) that uses a \textbf{binary heap} data structure to output the $n$-th strange number.  Explain why your algorithm is correct and analyze the run time of your algorithm. (Hint: Consider generating
    the strange numbers in increasing order, i.e., 2, 3, 4, 5, 6, 8, 9, 10, 12, 15, etc. Show how to generate efficiently the next strange
    number using a heap.)
\end{exercise}

\begin{solution}
    Begin with a min-heap with a root node of 1. At each step, pop the heap and
    continue popping until the root value changes. Then, insert $2r$, $3r$, and
    $5r$ into the the heap. Perform this $n$ times to determine the $n$-th
    strange number.

    Clearly, the heap will contain every strange number at some step, for if
    $2^{p_1}3^{p_2}5^{p_3}$ is some strange number, then, when the root is any
    of $2^{p_1-1}3^{p_2}5^{p_3}$, $2^{p_1}3^{p_2-1}5^{p_3}$, or
    $2^{p_1}3^{p_2}5^{p_3-1}$, we will push $2^{p_1}3^{p_2}5^{p_3}$. At each step,
    the root is the smallest strange number in the heap. Since duplicates are
    removed, the value returned is the $n$-th strange number. This argument can
    be formalized by induction.

    \begin{algorithm}[H]
        \caption{Outputs the $n$-th strange number.}
        \begin{algorithmic}[1]
        \Function{Strange-Number}{$n$}
            \State $H \gets \Call{Binary-Heap}{}$
            \State \(H.\textbf{push}(1)\)
            \ForRange{$i$}{1}{$n$}
            \State \(r\gets H.{\textbf{pop}}()\)\label{line:uglytarget}
            \While{\(H\) is nonempty and \(H[0]\equalto r\)} \Comment{Check for duplicate values of \(r\).}
            \State \(H.{\textbf{pop}}()\)\label{line:uglyduplicate}
            \EndWhile
            \State \(H.\textbf{push}\left(2r, 3r, 5r\right)\) \label{line:uglypush}
            \EndForRange
            \State \Return \(H[0]\)
        \EndFunction
    \end{algorithmic}
    \end{algorithm}
\end{solution}

\end{document}
